\documentclass[a4paper,12pt]{article}

\usepackage{fontspec}
\setmainfont{CMU Serif}
\setsansfont{CMU Sans Serif}
\setmonofont{CMU Typewriter Text}

\usepackage{polyglossia}
\setmainlanguage{russian}
\setotherlanguage{english}

\usepackage{graphicx}
\usepackage{geometry}
\geometry{left=2.5cm,right=2.5cm,top=2.5cm,bottom=2.5cm}

\usepackage{hyperref}

\usepackage{parskip}
\title{ChessDream : Платформа для Тренеров и Учителей Шахмат}
\author{Габдрахманов Азат}
\date{}

\begin{document}

\maketitle
\tableofcontents
\newpage

\section{Новая функциональность. Варианты}

\begin{itemize}
    \item \textbf{Уведомление участников клуба личесс в профиле о предстоящем турнире на личесс. Можно банально просто меню с предстоящими турнирами}
    \item \textbf{Загружать PGn и смотреть их}
    \item \textbf{По Pgn делать задачник ученику.}
    \item Еще идея на запоминание позиции.Например , выставляется позиция из 3-х фигур. Игрок запоминает.И через 30 секунд позиция исчезает.Игрок должен за минуту расставить правильно по памяти фигуры.
    \item Я на личесс , например, в школе ставлю ноутбук и они на время угадывают положение клеток. Детям нравится
    \item 4 = Дебют, чтобы ученик мог изучить варианты и сдать экзамен
5 =  Может файлы текстовые подгружать в кабинет ученику, чтобы он сам изучал
    \item Тогда нужно, чтобы ученик попадал в определённый клуб. И я смогу видеть.
Доступ в клуб сделать , чтобы я одобрял.
А лучше, чтобы тренер мог добавить ученика в клуб, а я одобрил эту заявку.
И тогда буду видеть всех, кто вступил в клуб.
Разделение на клубы для того, чтобы не получился хаос
\end{itemize}

\section{Новая функциональность. Итоговые варианты}

\begin{itemize}
    \item \textbf{Разделение на школы и клубы.} Школа состоит из произвольного количества клубов. У школы есть админ
    , этот админ может управлять деятельностью внутри клуба. В каждом клубе будет 1 или более тренеров.
    Ученик будет подавать заявку в школу, далее его админ определит в конкретный клуб к конкретному тренеру. Или это сделает руководитель отдельного клуба
    \item \textbf{Уведомление участников клуба личесс в профиле о предстоящем турнире на личесс. Предусмотреть возможность отправлять сообщения на личесс/tg/другие источники}
    \item \textbf{Тренер может загружать pgn файлы.} Каждый тренер может загружать pgn файлы лично к себе и использовать их приватно. Также каждый тренер может загрузить свой файл в папку отдельного клуба, где будут лежать все файлы клуба.
    Видеть файлы внутри клуба будут только тренеры клуба или приглашенные. Также будет отдельное файловое хранилище внутри школы, которое будет доступно всем тренерам.
    \item \textbf{Тренер может открывать свои файлы и анализировать/разыграывать их с учеником.} Возможно добавить туда движок.
    \item \textbf{По Pgn делать задачник ученику.} По формату файлов с дебютами делать ученику тренинг. Чтобы он делал ходы, а комп сам отвечал в соответствии с файлом.

\end{itemize}

\section{Новая функциональность. Потенциально полезные варианты}

\begin{itemize}

    \item Еще идея на запоминание позиции.Например , выставляется позиция из 3-х фигур. Игрок запоминает.И через 30 секунд позиция исчезает.Игрок должен за минуту расставить правильно по памяти фигуры.
    \item Я на личесс , например, в школе ставлю ноутбук и они на время угадывают положение клеток
    \item Онлайн подключение к доске тренера. Тренер открывает на сайте доску, и его ученик может присоединиться и смотреть.
    Тренер по ходу занятия может открывать  разные pgn файлы и показывать их

\end{itemize}


\end{document}
